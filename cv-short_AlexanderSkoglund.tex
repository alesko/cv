\documentclass[a4paper,margin,line]{res} \usepackage{latexsym}
\usepackage[latin1]{inputenc}
%\usepackage[swedish]{babel}

\oddsidemargin -.5in \evensidemargin -.5in \textwidth=6.0in
\addtolength{\textheight}{1.0in} \itemsep=0in \parsep=0in

\newenvironment{list1}{
  \begin{list}{\ding{113}}{%
      \setlength{\itemsep}{0in} \setlength{\parsep}{0in}
      \setlength{\parskip}{0in} \setlength{\topsep}{0in}
      \setlength{\partopsep}{0in}
      \setlength{\leftmargin}{0.17in}}}{\end{list}}
\newenvironment{list2}{
  \begin{list}{$\bullet$}{%
      \setlength{\itemsep}{0in} \setlength{\parsep}{0in}
      \setlength{\parskip}{0in} \setlength{\topsep}{0in}
      \setlength{\partopsep}{0in}
      \setlength{\leftmargin}{0.2in}}}{\end{list}}


\begin{document}
\name{Alexander Skoglund, Ph.D. (1976-02-28)}
\vspace*{.1in} 

\begin{resume}
\hspace{-34mm}
{\bf \large Senior Embedded Systems Engineer, self-employed at E$^2$ FirmwareLabs}
%\name{Mr. A. Nonymous, 1976-XX-XX \vspace*{.1in}}


\section{\sc Contact Information}
\vspace{.05in}
\begin{tabular}{@{}p{3in}p{3in}}
  Ranhammarsv\"agen 20 & {\it Mobile:} +46 (0)72-077 17 34 \\
  168 67 Bromma & {\it E-mail:} alexander.skoglund@e2labs.eu \\
  Sweden & {\it Homepage:} www.e2labs.eu \\
\end{tabular}


%\section{\sc Online profile}
%http://se.linkedin.com/in/alexanderskoglund

\section{\sc Experience Summary}

{\bf Methods}\\ {\em Scientific and engineering skills:} Robotics,
scientific methods, machine learning, computer vision, computer
graphics, scientific writing, real time programming, embedded
systems.\\ {\em Programming languages:} C and C++, {M}ATLAB (and GNU
Octave) and Python (basic user). \\ {\em Electronics:} System design,
analogue design, filter design, PCB design (prototype and production),
test and measurement, board bring-up.\\ {\em Work methods:} Agile
(scrum) and lean.

{\bf Tools}\\ {\em Software development:} GNU eco system (GCC for
ARM/AVR/MSP430, make, gdb, openocd, avrdude), Atmel Studio,
Microchip's XC compiler, IAR Workbench, CMake/Ninja, VSCode, Emacs, Eclipse
(multiple derivatives), Jenkins, Docker Unity and Robotframework for testing.
\\ {\em Host environment:} GNU/Linux since 2002. Several years of
experience with Apple's OSX.\\ {\em Version management:} Git
(Bitbucket, Github, Gerrit) and Subversion. \\ {\em Documenting:}
JIRA, \LaTeX, Doxygen and Wikis. \\ {\em Protocols:} I$^2$C, SPI,
UART, RS232, CAN, USB, TCP/IP sockets.\\ {\em Electronic design:}
KiCad, LTSpice, CadSoft Eagle, gerbview and surface mounted
soldering. \\ {\em Instrument:} Multimeter, oscilloscope, logical
analyser and protocol analyser.
%{\em Misc:} ROS, OpenGL, OpenCV, firewire.

%{\em Currently learning:} Code review methods and software test methods.

{\bf Microcontrollers and boards} ARM Cortex M0/M4/M7 (XMC44XX, STM32F4, 
STM32H7, Blue\-NRG), AVR (STK500, Arduino etc.), MSP430 (TI
LaunchPad), STM8 and Microchip dsPIC33EP.

{\bf Currently learning/improving:} Docker.
%ZephyrOS.
%Machine Learning using TinyML (Python) and  Rust.

%{\bf Software:} {\em Operating systems:} GNU/Linux; for software development and host (Ubuntu and Fedora) and Raspbian as an embedded OS.
%MacOS X; for some software development and desktop OS. {\bf Programming languages:} C and C++, {M}ATLAB (and GNU Octave),  Python (basic user).

%{\bf Hardware:} LTSpice for simulation, CadSoft Eagle for schematic capture and layout, analog design of Opamp, signal shaping for AD conversion, filters. Experience with surface mounted soldering.

%{\bf Misc:} Experience using OpenGL. ROS middleware (Robot Operating System, Willow Garage). {\bf Image acquisition:} firewire cameras and OpenCV.  Written several hardware drivers (e.g.,  air muscles, motion tracking, data acquisition). {\bf Protocols:} I2C, SPI and UART. Experience working in a Scrum team and JIRA.

%\section{\sc Present Position}

%{\bf Self employed at E2 Firmware Labs}, {\em Senior Embedded Systems Engineer}
%\hfill {\bf From 2020-03}\\

%\vspace{-.3cm}

\section{\sc Current Assignment}

{\bf Infinera}, Stockholm, Sweden\\
%\vspace{-5mm}
{\em Firmware engineer} \hfill {\bf From 2022-08}\\
Embedded software development for Infinera's fiber optical solution. 
Working with an ARM Cortex M4 that acts as an interface circuit. 
\\ {\bf Tools/techniques} used are; GCC toolchain, CMake, git, gdb, embedded C C++,
Google Test and Mock, grpc, docker and proprietary tools.

{\bf E$^2$ FirmwareLabs}, Stockholm, Sweden\\
%\vspace{-0mm}
{\em Hardware engineer} \hfill {\bf From 2021-04}\\
Doing PCB deign and layout for a client's project with primarily
analog components. Using JLC for prototype production. Maintain BOM
and work on sourcing components. Investigate new product design ideas
for ethernet connectivity.
\\ {\bf Tools/techniques} used is KiCad.

\vspace{3mm}

\section{\sc Professional Experience}

{\bf DigiSign}, Stockholm, Sweden\\
%\vspace{-5mm}
{\em Firmware engineer} \hfill {\bf 2022-03 -- 2022-08}\\
Helped the client porting a small proprietary realtime kernel from 
MIPS (different PICs) to ARM (STM32H7). The work is focused in low 
level programming close to the hardware; including interrupts, communication, 
scheduling, context-- and task switching, timers, memory management, etc.
\\ {\bf Tools/techniques} used are; GCC toolchain, gdb, Eclipse/STMCube, Make, git, 
ARM assembler, embedded C and proprietary tools.

\vspace{10mm}
{\bf Peratech}, Stockholm, Sweden\\
{\em Firmware engineer} \hfill {\bf 2021-11 -- 2022-02}\\
Helped the client with troubleshooting an ADC scanning device running
ZephyrOS. Worked on refactoring and merging for two codebases into one.\\ 
{\bf Tools/techniques} used are; ZephyrOS, GCC toolchain on
Linux, CMake, Make, git, embedded C.


{\bf Polarium}, Stockholm, Sweden\\
%\vspace{-.3cm}
{\em Senior Embedded Systems Engineer} \hfill {\bf 2021-06 -- 2021-10}\\
Working with firmware for Polarium's Battery Management System (BMS).
Investigate how a Hardware-In-the-Loop system can be built to automated
test of firmware. \\ {\bf Tools/techniques} used are; JIRA, git, Silicon Labs 8051
MCU, Subversion, IAR, embedded C.


{\bf DeLaval}, Stockholm, Sweden\\
%\vspace{-.3cm}
{\em Senior Embedded Systems Engineer} \hfill {\bf 2020-11 --
  2021-06}\\ At DeLaval I developed firmware for the next
generation of connected devices. Provisioning of IoT devices, MQTT
communication between embedded device/PLC and AWS
cloud. \\ {\bf Tools/techniques} used are; JIRA, git, GCC toolchain
(including Cmake/make) on Linux, AWS IoT Core and AWS infrastructure,
embedded C, ARM Cortex microcontroller, FreeRTOS, SSL/TLS encrypted
MQTT communication, PLC programming and Wireshark.


{\bf InMotion}, Stockholm, Sweden\\
%\vspace{-.3cm}
{\em Senior Embedded Systems Engineer} \hfill {\bf 2020-03 --
  2020-11}\\ My main work was with firmware requirements,
implementation, testing and verification. Main area was Functional
Safety of inverters for a motor control system. Firmware must fulfill
automotive standards (e.g., ISO 26262 requirements). The team worked
according to Scrum. \\ {\bf Tools/techniques} used; JIRA, Crucible, Jenkins,
Subversion, IAR, embedded C, unit testing, gcov, Python and Robot
Framework.

{\bf Scanreco}, Stockholm, Sweden\\
%\vspace{-.3cm}
{\em Senior Embedded Systems Engineer} \hfill {\bf 2019-04 --
  2020-03}\\ Working with firmware for the next generation of
Scanreco's professional remote controls. Primarily working with the
Bluetooth connectivity for firmware upgrades (bootloaders) and remote
configuration via the back-end system. he team worked according to
Scrum. \\ {\bf Tools/techniques} used are; GCC toolchain on Linux, Make, git,
embedded C, Bluetooth Low Energy (ST 's BlueNRG-2 SoC; ARM Cortex
microcontroller), unit testing (C Unit), IAR, STM8.

{\bf Realtime Embedded}, Stockholm, Sweden\\
%\vspace{-.3cm}
{\em Firmware/hardware engineer} \hfill {\bf 2017-03 --
  2019-03}\\ Working with electronics and firmware design for RTE's
clients. Primarily working on system design and firmware for hardware
interaction, and also electronic development and board bring-up. The
largest project (running for 18 months) involved power control of an
inverter with CAN communication and bootloading (among others).

{\bf BioServo Technologies}, Stockholm, Sweden\\
%\vspace{-.3cm}
{\em Firmware/hardware engineer} \hfill {\bf 2014-02 --
  2017-02}\\ Working on electronic and firmware design on BioServo's
SEM glove and next generation of BioServo's products.  Primarily
working on electronic development, board bring-up, system design and
firmware for hardware interaction. Responsible for software in a
scientific project to evaluate BioServo's product.  Also maintaining
current production version with upgrades and production support.
%{\bf Tools:} Cadsoft Eagle (PCB design), LTSpice (electronic
%simulation), GCC toolchain (programming) for ARM and AVR.


{\bf {\AA}F Group}, Stockholm, Sweden
%\vspace{-.3cm}
{\em Embedded Software Eng.} \hfill {\bf 2012-11 -- 2014-02}\\ Working
on firmware design for client's embedded systems.  {\em Project 1:}
embedded Linux on ARM for an autonomous mobile robot.  {\em Project
  2:} embedded Linux web server on a RaspberryPi, short investigation
and demo mock-up.  {\em Project 3:} PWM control using an MSP430, short
investigation.  {\em Project~4:} control of an electrochromic foil
(AVR) and powerline communication (LonTalk), 9 months.

%\vspace{.5cm}

{\bf Karolinska Institute}, {\em Research Engineer}, Stockholm, Sweden
\hfill {\bf 2009-12 -- 2012-11}\\ Research Engineer in Brain, Body \&
Self Laboratory (Ehrsson group), at the Department of Neuroscience.
My main task was to develop custom electronics and software for
research in neuroscience.  I worked on developing an MR compatible
robotic anthropomorphic hand for neuroscience research on body
perception, body ownership, agency, and prosthesis.

{\bf \"Orebro University}, {\em Research Assistant}, \"Orebro, Sweden
\hfill {\bf 2009-06 -- 2009-12}\\ Work on modelling of human grasping
strategies. Modelled grasping skills should be possible to transfer to
dexterous robotic hand. This work is part of the HANDLE project,
funded by EU:s FP7.
%I also gave a master course on ``Introduction on Robotics and Intelligent Systems''.

{\bf \"Orebro University}, {\em Ph.D Student}, \"Orebro, Sweden \hfill
{\bf Nov, 2003 -- June, 2009}\\ Research on
Programming-by-Demonstration including learning systems, human-machine
interfaces, human-like motions.  In addition, I have been a teaching
assistant in several courses.  On parental leave in 2008-02--2008-05
and 2008-12--2009-01 (50\%).

{\bf \"Orebro University}, {\em Research engineer}, \"Orebro, Sweden
\hfill {\bf June, 2001 -- Nov, 2003}\\ Maintenance with mobile robots
and lightweight manipulators. Also, involved in the electronic design
of an electronic tongue for water quality assessment, control system
for manipulators, range finding sonar for manipulation.

{\bf Aerotech Telub}, {\em Radio engineer}, Arboga, Sweden \hfill {\bf
  Jan, 2000 -- June, 2001}\\ Carried out several consulting projects
associated with a short range radio for the take off team around the
fighter aircraft SAAB JAS Gripen.  I was involved in preliminary radio
measurements for installation of a radio system in the metro of
Barcelona.

%\vspace{-.3cm}
\section{\sc Courses}
{\bf Non academic courses}\\
\begin{list2}
\item Nov. 26--27, 2019 Machine safety
\item Maj 16, 2019 How to Develop Better Firmware Faster by Jack Ganssle
  \item Apr. 23--24, 2013 Building the IoT with Thingsquare Mist and Contiki
\end{list2}


\section{\sc Education}
{\bf \"Orebro University}, \"Orebro, Sweden\\

\vspace*{-.1in}
\begin{list2}
\item Ph.D., Computer Science, {\it Programming by Demonstration of
  Robot Manipulators} 2009.
\vspace*{.03in}
%\item[] Licentiate, {\it Towards Manipulator Learning by Demonstration and Reinforcement Learning}, 2006.
%\vspace*{.03in}
\item M.Sc., Electrical and Electronic Engineering, December 2002.
\vspace*{.03in}
\item B.A., Electrical Engineering, September 1998.
\end{list2}

%\section{\sc Research during Ph.D. studies}
\section{\sc Academic skills}
During my Ph.D. my research field was within imitation learning, where
a robot learns a task by observing a human.  Learning should then
continue to improve the performance by further self-observation or by
providing more knowledge from the demonstrator (teacher). The main
application is to simplify the programming process of an industrial
manipulator (robot arm), a.k.a. ``Programming-by-Demonstration''.
%Top Three Publications:
%Within this framework the following research areas are of interest: learning systems,
%human-machine interfaces, human-like motions.


%\section{\sc Top Three Publications}

%Alexander Skoglund, Boyko Iliev and Rainer Palm.  {\it
%  Programming-by-Demonstration of Reaching Motions -- A
%  Next-State-Planner Approach} Robotics and Autonomous Systems. Volume
%58, Issue 5, 31 May 2010,

%Alexander Skoglund, Johan Tegin, Boyko Iliev and Rainer Palm.  {\it
%  Programming-by-Demonstration of Reaching Motions for Robot
%  Grasping}, Presented at the 14th International Conference on
%Advanced Robotics (ICAR 2009) Germany, Munich, June 22th to 26th,
%2009.

%Alexander Skoglund, Boyko Iliev and Rainer Palm.  {\it A Hand State
%  Approach to Imitation with a Next-State-Planner for Industrial
%  Manipulators} Presented at 2008 International Conference on
%Cognitive Systems Karlsruhe, Germany, April 2-4.

%\begin{list2}
%\item Alexander Skoglund, Boyko Iliev and Rainer Palm.  {\it
%  Programming-by-Demonstration of Reaching Motions -- A
%  Next-State-Planner Approach} Robotics and Autonomous Systems. Volume
%58, Issue 5, 31 May 2010,

%\item Alexander Skoglund, Johan Tegin, Boyko Iliev and Rainer Palm.  {\it
%  Programming-by-Demonstration of Reaching Motions for Robot
%  Grasping}, Presented at the 14th International Conference on
%Advanced Robotics (ICAR 2009) Germany, Munich, June 22nd to 26th,
%2009.

%\item Alexander Skoglund, Boyko Iliev and Rainer Palm.  {\it A Hand State
%%  Approach to Imitation with a Next-State-Planner for Industrial
%  Manipulators} Presented at 2008 International Conference on
%Cognitive Systems Karlsruhe, Germany, April 2-4.
%\end{list2}



%\section{\sc Reviewer}


%{\it IEEE Robotics and Automation Magazine}, {\it IEEE Transactions on
%  Automation Sciences and Engineering} and {\it Neural Computing and
% Applications}.  Reviewer for several conferences, including:

%\begin{list2}
%\item IEEE/RSJ International Conference on Intelligent Robots and
%  Systems: 2011, 2010 and 2007
%\item IEEE International Conference on Robotics and Automation (ICRA)
%  2009 and 2006.
%\end{list2}

%\section{\sc Supervision}

%I have supervised two master theses:
%\begin{list2}
%\item Sofia Cruz-Ferreira Fr\"oman, master thesis: {\it An actuated
%  rubber hand for use in MRI environments}. KTH (Royal Institute of
%  Technology, Stockholm).
%\item Guiseppe Valerio's master thesis on a medical diagnostic
%  application. Chalmers University of Technology, Göteborg.
%\end{list2}

%\section{\sc Pedagogical Skills}

{\em Teaching} \hfill {\bf August 2004 - December
  2009}\\ Class: Introduction to Robotics and Intelligent Systems, Master
Course. Autumn 2009. An introductory course in robotics. Topics
covered: Robotic history, actuators, manipulation, sensing and
perception, localisation, navigation, mapping, state estimation, dead
reckoning, Bayesian filters and multi robot applications.
%{\em Teaching Assistant} \hfill {\bf August 2004 - January 2008}
\\ Duties at various times leading weekly computer lab exercises:

\vspace*{.05in}
\begin{list2}
\item TDD121/PRG045 Programming in C, Winter 2006 and 2007.
\item TDM136 Methods for Modelling, Simulation and Visualisation, Fall
  2007.
%\item TDD112 Web Client Programming, Autumn 2006.
\item TDD112 Computer Graphics, Fall and Autumn 2004, Autumn 2005,
  Fall 2006, Fall 2007.
%\item Introduction to Technology, Fall, 2004.
\end{list2}



%{\em Popular Science Article}\\ Alexander Skoglund and Boyko Iliev.
%{\it Programming By Demonstrating -- Robots Task Primitives} SERVO
%Magazine, December 2007.

%\section{\sc Ph.D. Courses}
%Soft Computing for Control,
%Information search and information resources,
%Research Methodology,
%System Identification and Forecasting by Neural Networks,
%Artificial Neural Networks,
%Real Time Programming,
%Computer Graphics,
%Reinforcement Learning,
%Machine Learning,
%Introduction to Robotics,
%Kinematics and Dynamics,
%AI Robotics,
%Popular Science Writing,
%Computational Motor Learning,
%Parallel Programming.

%\section{\sc Master thesis}
%Degree Project in Electrical and Electronic Engineering: {\em ``Implementing a microcontroller on a radio controller board''}.


\end{resume}
\end{document}
